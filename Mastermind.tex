\documentclass[]{article}
\usepackage[utf8]{inputenc}
\usepackage{amsthm}
\usepackage{amsfonts}

%opening
\title{Résolution du Mastermind}
\author{Clara Bringer, Pierre Gervais}

\newtheorem{mydef}{Définition}
\newtheorem{mythm}{Théorème}
\newtheorem{mycoro}{Corollaire}

\begin{document}

\maketitle

\part*{Introduction}

\begin{mydef}
	Placement : Une certaine couleur à une certaine position.
\end{mydef}

\begin{mydef}
	$N_f \in \mathbb{N}$ désigne le nombre de fiches et $N_c \in \mathbb{N}$ le nombre de couleurs.
\end{mydef}

\part*{Raisonnements et conclusions}

\section{Raisonnement ligne par ligne}

\begin{mythm}
Si sur une ligne on n'a aucun témoin noir, alors chaque couleur est une couleur impossible à sa position respective.
\end{mythm}

\begin{mythm}
	Si sur une ligne on n'a aucun témoin, alors aucune des couleurs n'apparaît dans la solution.
\end{mythm}

\begin{mythm}[A vérifier !]
	Si deux fiches ont les mêmes couleurs impossibles, alors elles sont dites équivalentes, c'est à dire que tester une couleur sur la première ou la seconde n'est pas plus cohérent l'un que l'autre.
\end{mythm}

\section{Raisonnement comparatif}

\subsection{Comparaison avec la solution partielle}

\begin{mythm}
Soit une ligne avec $n$ placements finaux et $n$ témoins noirs, alors les autres placements de cette ligne sont impossibles.
\end{mythm}

\begin{mythm}
On a pour une position $N_c-1$ couleurs impossibles si et seulement si la couleur restante est la couleur de ce placement.
\end{mythm}

\begin{mythm}
Si une couleur est présente dans la solution et qu'elle est impossible à $N_f-1$ positions, c'est à dire qu'elle n'est possible qu'à une unique position, alors c'est qu'elle est la couleur du placement à cette position.
\end{mythm}

\subsection{Comparaison entre lignes}

\begin{mythm}
Soit $L_a$ une ligne contenant, entre autres, des pions de couleurs $(C_i)$.
\\
Soit une autre ligne $L_b$, sur laquelle \textbf{seulement} $m$ de ces pions ont été remplacés par des pions de couleurs différentes de $(C_i)$, que l'on notera $(C'_i)$, à présent en quantités $(n'_i)$ sur $L_b$. Les couleurs $(C_i)$, elles, sont en quantités $(n_i)$.
\\Si $L_b$ possède $m$ témoins de plus, alors :
\begin{itemize}
	\item Les quantités $(n_i)$ majorent les quantités des couleurs $(C_i)$ de la solution.
	\item Les quantités $(n'_i)$ minorent les quantités des couleurs $(C'_i)$ de la solution.
\end{itemize}
\end{mythm}

\begin{proof}[Démonstration]
	Commençons par traiter les couleurs $(C_i)$ :
	\\
	Par l'absurde, supposons qu'il existe un $n_i$ strictement inférieur à la quantité de couleur $C_i$ de la solution.
	\\
	Alors, il y a sur $L_b$ au moins un pion de couleur $C_i$ de moins que dans la solution, celui ci faisait partie des $m$ pions considérés sur $L_a$, ainsi son retrait fera qu'un des témoins de $L_a$ disparaîtra.
	\\
	On sait que de $L_a$ à $L_b$ il y a $m$ pions qui diffèrent, dont un de couleur $C_i$ qui était présent dans la solution, alors le nombre de nouveaux témoins de $L_a$ à $L_b$ est au plus $m - 1$.
	\\
	On est donc en contradiction avec l'hypothèse des $m$ témoins en plus, on a donc démontré que les quantités $(n_i)$ majorent celles des couleurs $(C_i)$ dans la solution.
	\\
	A présent les couleurs $(C'_i)$ :
	\\
	Par l'absurde, supposons qu'il existent un $n'_i$ strictement supérieur à la quantité de couleur $C'_i$ de la solution.
	\\
	Alors, il y a au moins un pion de couleur $C'_i$ de plus que dans la solution, celui-ci faisait partie des $m$ pions qui ont changé de $L_a$ à $L_b$ et comme il n'appartient pas à la solution, son ajout ne modifiera pas le nombre de témoins.
	\\
	Ainsi le nombre de nouveaux témoins est au plus $m-1$, ce qui contredit l'hypothèse des $m$ témoins en plus.
\end{proof}

\begin{mycoro}
Si on a déjà découvert $n_i$ placements de couleur $C_i$, alors on a découvert tous les placements de cette couleur ; c'est donc une couleur impossible sur les autres fiches.
\end{mycoro}

\begin{mycoro}
Si on a une quantité $n_i=0$, alors la couleur $C_i$ n'est pas dans la solution.
\end{mycoro}

\begin{mycoro}
S'il existe une quantité $n'_i$ telle que l'on connaisse $N_f - n'_i$ placements finaux \textbf{de couleurs différentes de $C'_i$}, alors les placements restants sont de couleurs $C'_i$.
\end{mycoro}

\begin{mythm}
	Soit une ligne $L_a$ et une ligne $L_b$ qui diffèrent de $n$ placements.
	\begin{itemize}
		\item Si on a $n$ témoins noirs de plus, alors les nouveaux placements appartiennent à la solution.
		\item Si on a $n$ témoins noirs de moins, alors les anciens placements appartiennent à la solution.
	\end{itemize} 
\end{mythm}

\part*{Algorithme de résolution}

\paragraph*{}
Cet algorithme exploite les théorèmes énoncés dans la partie précédente, le principe est d'évaluer à chaque pas quelle combinaison (cohérente), une fois soumise et les témoins donnés, éliminerait le plus de placements.
\\
On garde en mémoire les couleurs impossibles pour une certaine fiche et un encadrement des quantités finales de chaque couleur du jeu (initialement le couple $(0, N_f)$)
\\
Tant que l'on n'a pas trouvé la solution :\\
	$\rightarrow$Pour chaque combinaison cohérente avec les déductions précédentes :\\
		$\rightarrow \rightarrow$Pour chaque $N_f$-uplet de témoins possible :\\
			$\rightarrow \rightarrow \rightarrow$Déduire à partir des déductions précédentes les placements impossibles.\\
		$\rightarrow \rightarrow$Faire la moyenne des placements impossibles pour cette combinaison.\\
	$\rightarrow$Soumettre la combinaison avec la moyenne la plus élevée.\\
	$\rightarrow$En déduire les placements impossibles.\\

\end{document}
